%%%%%%%%%%%%%%%%%%%%%%%%%%%%%%%%%%%%%%%%%%%%%%%%%%%%%%%%%%%%%%%%%%%%%%
% How to use writeLaTeX: 
%
% You edit the source code here on the left, and the preview on the
% right shows you the result within a few seconds.
%
% Bookmark this page and share the URL with your co-authors. They can
% edit at the same time!
%
% You can upload figures, bibliographies, custom classes and
% styles using the files menu.
%
% If you're new to LaTeX, the wikibook is a great place to start:
% http://en.wikibooks.org/wiki/LaTeX
%
%%%%%%%%%%%%%%%%%%%%%%%%%%%%%%%%%%%%%%%%%%%%%%%%%%%%%%%%%%%%%%%%%%%%%%
\documentclass{tufte-book}

\hypersetup{colorlinks}% uncomment this line if you prefer colored hyperlinks (e.g., for onscreen viewing)

%%
% Book metadata
\title{An introduction to generalized complex geometry\thanks{Thanks to Edward R.~Tufte for his inspiration.}}
\author[Lennart Döppenschmitt]{Lennart Döppenschmitt}
\publisher{The University of Zurich}

%%
% If they're installed, use Bergamo and Chantilly from www.fontsite.com.
% They're clones of Bembo and Gill Sans, respectively.
%\IfFileExists{bergamo.sty}{\usepackage[osf]{bergamo}}{}% Bembo
%\IfFileExists{chantill.sty}{\usepackage{chantill}}{}% Gill Sans

%\usepackage{microtype}

%% math libraries
\usepackage{amsmath, amsthm, amssymb}
\usepackage{thmtools}
\usepackage{latex/pkg/math}
\usepackage{silence}
% \WarningFilter{latex}
% \WarningFilter{latex}{You have requested package}



%%
% For nicely typeset tabular material
\usepackage{booktabs}

%%
% For graphics / images
\usepackage{graphicx}
\setkeys{Gin}{width=\linewidth,totalheight=\textheight,keepaspectratio}
\graphicspath{{graphics/}}

% The fancyvrb package lets us customize the formatting of verbatim
% environments.  We use a slightly smaller font.
\usepackage{fancyvrb}
\fvset{fontsize=\normalsize}

%%
% Prints argument within hanging parentheses (i.e., parentheses that take
% up no horizontal space).  Useful in tabular environments.
\newcommand{\hangp}[1]{\makebox[0pt][r]{(}#1\makebox[0pt][l]{)}}

%%
% Prints an asterisk that takes up no horizontal space.
% Useful in tabular environments.
\newcommand{\hangstar}{\makebox[0pt][l]{*}}






% new environments
\newcounter{example}[section]
\newenvironment{example}[1][]{\refstepcounter{example}\par\medskip
   \noindent \textbf{Example~\theexample. #1} \rmfamily}{\medskip}

\newenvironment{definition}[1][]{\refstepcounter{example}\par\medskip
   \noindent \textbf{Definition~\theexample. #1} \rmfamily}{\medskip}

\newenvironment{lemma}[1][]{\refstepcounter{example}\par\medskip
   \noindent \textbf{Lemma~\theexample. #1} \rmfamily}{\medskip}

% \newcounter{remark}[section]
\newenvironment{remark}[1][]{\refstepcounter{example}\par\medskip
   \noindent \textbf{Remark~\theexample. #1} \rmfamily}{\medskip}

% \newcounter{theorem}[section]
\newenvironment{theorem}[1][]{\refstepcounter{example}\par\medskip
   \noindent \textbf{Theorem~\theexample. #1} \rmfamily}{\medskip}

% \newcounter{exercise}[section]
\newenvironment{exercise}[1][]{\refstepcounter{example}\par\medskip
   \noindent \textbf{Exercise~\theexample. #1} \rmfamily}{\medskip}

% Better spacing than regular itemize
\newenvironment{itemize*}
  {\begin{itemize}[topsep=-\parskip+\jot,itemsep=-\parskip-\jot]}
  {\end{itemize}}

% Better spacing than regular enumerate, (a), (b), ...
\newenvironment{enumerate*}
  {\begin{enumerate}[label=(\alph*),topsep=-\parskip+\jot,itemsep=-\parskip-\jot]}
  {\end{enumerate}}

% Better spacing than regular enumerate, (i), (ii), ...
\newenvironment{enumerate**}
  {\begin{enumerate}[label=(\roman*),topsep=-\parskip+\jot,itemsep=-\parskip-\jot]}
  {\end{enumerate}}

% Better spacing than regular enumerate, (a'), (b'), ...
\newenvironment{enumerate***}
  {\begin{enumerate}[label=(\alph*'),topsep=-\parskip+\jot,itemsep=-\parskip-\jot]}
  {\end{enumerate}}




%%
% Prints a trailing space in a smart way.
\usepackage{xspace}

%%
% Some shortcuts for Tufte's book titles.  The lowercase commands will
% produce the initials of the book title in italics.  The all-caps commands
% will print out the full title of the book in italics.
\newcommand{\vdqi}{\textit{VDQI}\xspace}
\newcommand{\ei}{\textit{EI}\xspace}
\newcommand{\ve}{\textit{VE}\xspace}
\newcommand{\be}{\textit{BE}\xspace}
\newcommand{\VDQI}{\textit{The Visual Display of Quantitative Information}\xspace}
\newcommand{\EI}{\textit{Envisioning Information}\xspace}
\newcommand{\VE}{\textit{Visual Explanations}\xspace}
\newcommand{\BE}{\textit{Beautiful Evidence}\xspace}

\newcommand{\TL}{Tufte-\LaTeX\xspace}

% Prints the month name (e.g., January) and the year (e.g., 2008)
\newcommand{\monthyear}{%
  \ifcase\month\or January\or February\or March\or April\or May\or June\or
  July\or August\or September\or October\or November\or
  December\fi\space\number\year
}


% Prints an epigraph and speaker in sans serif, all-caps type.
\newcommand{\openepigraph}[2]{%
  %\sffamily\fontsize{14}{16}\selectfont
  \begin{fullwidth}
  \sffamily\large
  \begin{doublespace}
  \noindent\allcaps{#1}\\% epigraph
  \noindent\allcaps{#2}% author
  \end{doublespace}
  \end{fullwidth}
}

% Inserts a blank page
\newcommand{\blankpage}{\newpage\hbox{}\thispagestyle{empty}\newpage}

\usepackage{units}

% Typesets the font size, leading, and measure in the form of 10/12x26 pc.
\newcommand{\measure}[3]{#1/#2$\times$\unit[#3]{pc}}

% Macros for typesetting the documentation
\newcommand{\hlred}[1]{\textcolor{Maroon}{#1}}% prints in red
\newcommand{\hangleft}[1]{\makebox[0pt][r]{#1}}
\newcommand{\hairsp}{\hspace{1pt}}% hair space
\newcommand{\hquad}{\hskip0.5em\relax}% half quad space
\newcommand{\TODO}{\textcolor{red}{\bf TODO!}\xspace}
\newcommand{\ie}{\textit{i.\hairsp{}e.}\xspace}
\newcommand{\eg}{\textit{e.\hairsp{}g.}\xspace}
\newcommand{\na}{\quad--}% used in tables for N/A cells
\providecommand{\XeLaTeX}{X\lower.5ex\hbox{\kern-0.15em\reflectbox{E}}\kern-0.1em\LaTeX}
\newcommand{\tXeLaTeX}{\XeLaTeX\index{XeLaTeX@\protect\XeLaTeX}}
% \index{\texttt{\textbackslash xyz}@\hangleft{\texttt{\textbackslash}}\texttt{xyz}}
\newcommand{\tuftebs}{\symbol{'134}}% a backslash in tt type in OT1/T1
\newcommand{\doccmdnoindex}[2][]{\texttt{\tuftebs#2}}% command name -- adds backslash automatically (and doesn't add cmd to the index)
\newcommand{\doccmddef}[2][]{%
  \hlred{\texttt{\tuftebs#2}}\label{cmd:#2}%
  \ifthenelse{\isempty{#1}}%
    {% add the command to the index
      \index{#2 command@\protect\hangleft{\texttt{\tuftebs}}\texttt{#2}}% command name
    }%
    {% add the command and package to the index
      \index{#2 command@\protect\hangleft{\texttt{\tuftebs}}\texttt{#2} (\texttt{#1} package)}% command name
      \index{#1 package@\texttt{#1} package}\index{packages!#1@\texttt{#1}}% package name
    }%
}% command name -- adds backslash automatically
\newcommand{\doccmd}[2][]{%
  \texttt{\tuftebs#2}%
  \ifthenelse{\isempty{#1}}%
    {% add the command to the index
      \index{#2 command@\protect\hangleft{\texttt{\tuftebs}}\texttt{#2}}% command name
    }%
    {% add the command and package to the index
      \index{#2 command@\protect\hangleft{\texttt{\tuftebs}}\texttt{#2} (\texttt{#1} package)}% command name
      \index{#1 package@\texttt{#1} package}\index{packages!#1@\texttt{#1}}% package name
    }%
}% command name -- adds backslash automatically
\newcommand{\docopt}[1]{\ensuremath{\langle}\textrm{\textit{#1}}\ensuremath{\rangle}}% optional command argument
\newcommand{\docarg}[1]{\textrm{\textit{#1}}}% (required) command argument
\newenvironment{docspec}{\begin{quotation}\ttfamily\parskip0pt\parindent0pt\ignorespaces}{\end{quotation}}% command specification environment
\newcommand{\docenv}[1]{\texttt{#1}\index{#1 environment@\texttt{#1} environment}\index{environments!#1@\texttt{#1}}}% environment name
\newcommand{\docenvdef}[1]{\hlred{\texttt{#1}}\label{env:#1}\index{#1 environment@\texttt{#1} environment}\index{environments!#1@\texttt{#1}}}% environment name
\newcommand{\docpkg}[1]{\texttt{#1}\index{#1 package@\texttt{#1} package}\index{packages!#1@\texttt{#1}}}% package name
\newcommand{\doccls}[1]{\texttt{#1}}% document class name
\newcommand{\docclsopt}[1]{\texttt{#1}\index{#1 class option@\texttt{#1} class option}\index{class options!#1@\texttt{#1}}}% document class option name
\newcommand{\docclsoptdef}[1]{\hlred{\texttt{#1}}\label{clsopt:#1}\index{#1 class option@\texttt{#1} class option}\index{class options!#1@\texttt{#1}}}% document class option name defined
\newcommand{\docmsg}[2]{\bigskip\begin{fullwidth}\noindent\ttfamily#1\end{fullwidth}\medskip\par\noindent#2}
\newcommand{\docfilehook}[2]{\texttt{#1}\index{file hooks!#2}\index{#1@\texttt{#1}}}
\newcommand{\doccounter}[1]{\texttt{#1}\index{#1 counter@\texttt{#1} counter}}


\DeclareDocumentCommand{\vpv}{}{V \oplus V^{*}}
\DeclareDocumentCommand{\V}{}{\mathbb{V}}
\DeclareDocumentCommand{\so}{}{\mathfrak{so}}
\DeclareDocumentCommand{\a}{}{\mathfrak{a}}
\DeclareDocumentCommand{\SO}{}{SO}

% Generates the index
\usepackage{makeidx}
\makeindex

\begin{document}

% Front matter
\frontmatter

% r.1 blank page
% \blankpage

% v.2 epigraphs
% \newpage\thispagestyle{empty}
% \openepigraph{%
% The public is more familiar with bad design than good design.
% It is, in effect, conditioned to prefer bad design, 
% because that is what it lives with. 
% The new becomes threatening, the old reassuring.
% }{Paul Rand%, {\itshape Design, Form, and Chaos}
% }
% \vfill
% \openepigraph{%
% A designer knows that he has achieved perfection 
% not when there is nothing left to add, 
% but when there is nothing left to take away.
% }{Antoine de Saint-Exup\'{e}ry}
% \vfill
% \openepigraph{%
% \ldots the designer of a new system must not only be the implementor and the first 
% large-scale user; the designer should also write the first user manual\ldots 
% If I had not participated fully in all these activities, 
% literally hundreds of improvements would never have been made, 
% because I would never have thought of them or perceived 
% why they were important.
% }{Donald E. Knuth}


% r.3 full title page
\maketitle


% v.4 copyright page
% \newpage
% \begin{fullwidth}
% ~\vfill
% \thispagestyle{empty}
% \setlength{\parindent}{0pt}
% \setlength{\parskip}{\baselineskip}
% Copyright \copyright\ \the\year\ \thanklessauthor

% \par\smallcaps{Published by \thanklesspublisher}

% \par\smallcaps{tufte-latex.googlecode.com}

% \par Licensed under the Apache License, Version 2.0 (the ``License''); you may not
% use this file except in compliance with the License. You may obtain a copy
% of the License at \url{http://www.apache.org/licenses/LICENSE-2.0}. Unless
% required by applicable law or agreed to in writing, software distributed
% under the License is distributed on an \smallcaps{``AS IS'' BASIS, WITHOUT
% WARRANTIES OR CONDITIONS OF ANY KIND}, either express or implied. See the
% License for the specific language governing permissions and limitations
% under the License.\index{license}

% \par\textit{First printing, \monthyear}
% \end{fullwidth}

% r.5 contents
\tableofcontents

% \listoffigures

% \listoftables

% r.7 dedication
% \cleardoublepage
% ~\vfill
% \begin{doublespace}
% \noindent\fontsize{18}{22}\selectfont\itshape
% \nohyphenation
% Dedicated to those who appreciate \LaTeX{} 
% and the work of \mbox{Edward R.~Tufte} 
% and \mbox{Donald E.~Knuth}.
% \end{doublespace}
% \vfill
% \vfill





%%%%%%%%%%%%%%%%%%%%%%%%%%%%%%%%%%%%%%%%%%%%%%%%%%%%%%%%%%%%%%%%%%%%%%%%%%%%%%%%
%%%%%%%%%%%%%%%%%%%%%%%%%%%%%%%%%%%%%%%%%%%%%%%%%%%%%%%%%%%%%%%%%%%%%%%%%%%%%%%%
%%%%%%%%%%%%%%%%%%%%%%%%%%%%%%%%%%%%%%%%%%%%%%%%%%%%%%%%%%%%%%%%%%%%%%%%%%%%%%%%
\chapter*{Introduction}


These notes grew out of an introductory course on generalized complex geometry taught
in the fall of 2022 at the University of Zurich. None of the content is original to the
author of these notes. The author would like to thank 

\lb
This course is far from a complete account of the subject, instead the aim is to
get a quick overview and work towards essential milestones in the development of generalied
geometry.


\lb
{\bf Generalized reduction.}
Review symplectic and holomorphic reduction.
\begin{enumerate}
    \item stienonXu2007
    \item zambon2008 generalization of coisotropic reduction
    \item bursztynCavalcantiGualtieri2006
\end{enumerate}
There are no real punch line examples. What about the GC structure on the instanton moduli
space? from bursztynCavalcantiGualtieri2013?



\lb
{\bf Generalized Kähler structures?}
This really goes into sigma models and super symmetry as examples.
I know a lot of about this topic, that would make it easier to prepare


\lb
Branes in prequantization and physics?
\begin{enumerate}
    \item kapustinOrlov
\end{enumerate}



\lb
{ \bf GCG and supersymmetry?}
kapustinLi2005 for a discussion of sigma models, supersymmetry and generalized
complex structures.
This could also lead to T duality?






% This sample book discusses the design of Edward Tufte's
% books\cite{Tufte2001,Tufte1990,Tufte1997,Tufte2006}
% and the use of the \doccls{tufte-book} and \doccls{tufte-handout} document classes.


%%
% Start the main matter (normal chapters)
% \mainmatter














\chapter{Geometric structures from differential forms}
\label{ch:ZT1VUV}
\marginnote{ZT1VUV}

To revisit a few classical geometric structures and become accustomed with
notation and conventions, we will dive into geometric structures provided by differentia forms
and explore their integrability conditions.
\footnotetext{This contains Poisson geometry. This is a side note}

\section{Background}

\newthought{We will assume} throughout this course that $M$ is a smooth manifold. \index{manifold}
Whenever we use a category, it will be in a \texttt{monospaced} font, like so.
\sidenote{this is a side note about categories.}
What about other equations?
$$\sum _{i = 0} i^{2}$$
\marginnote{This is a rather unconventional summation.}
This is supposed to be \hlred{red}.


\begin{figure}
    \centering
    \begin{tikzpicture}[scale=1.3]
    \node (m) at (0,0) {$(M, Q)$};
    \node (s) at (1,1) {$(S, ω)$};
    \node (n) at (2,0) {$(N, P)$};
    \draw[->](s) -- node[above left] {$π_{M}$} (m);
    \draw[->](s) -- node[above right] {$π_{N}$} (n);
    \end{tikzpicture}
    \caption{This is a Poisson span}
    \label{fig:NIYJS9}
\end{figure}










\section{Foliations}





\subsection{Reeb Foliation}






\section{Reduction of structure group}

A very popular and powerful perspective on geometric structures is the reductio of structure
groups. 

\marginnote{compare from shaffhauser2009 Proposition 2.3}

\lb
An oriented Riemannian $n$-manifold $(M, g)$ corresponds to a reduction to $SO(n)$.

\lb
A complex structure is a reduction from $GL_{n}$ to $U(n)$.





%%%%%%%%%%%%%%%%%%%%%%%%%%%%%%%%%%%%%%%%%%%%%%%%%%%%%%%%%%%%%%%%%%%%%%%%%%%%%%%%
%%%%%%%%%%%%%%%%%%%%%%%%%%%%%%%%%%%%%%%%%%%%%%%%%%%%%%%%%%%%%%%%%%%%%%%%%%%%%%%%
%%%%%%%%%%%%%%%%%%%%%%%%%%%%%%%%%%%%%%%%%%%%%%%%%%%%%%%%%%%%%%%%%%%%%%%%%%%%%%%%
\chapter{$V \oplus V^{*}$}


\lb
We're using \cite{gualtieri2004},





%%%%%%%%%%%%%%%%%%%%%%%%%%%%%%%%%%%%%%%%%%%%%%%%%%%%%%%%%%%%%%%%%%%%%%%%%%%%%%%%
\section{Linear Algebra}

\lb
Let $V$ be a real vector space of dimension $n$. This chapter is devoted to the Linear algebra of
the vector space $V \oplus V^{*}$.
\marginnote{marcut0000}

\lb
$V \oplus V^{*}$ always carries a bilinear, symmetric pairing
$$\ab{X + ξ, Y + η} = ½ ( ξ(Y) + η(X))$$
for $X, Y ∈ V$ and $ξ, η ∈ V^{*}$.
Let $e_{1}, \ldots e_{n}$ be a basis of $V$ and $e^{1}, \ldots, e^{n}$ its dual basis. We choose
their union as a basis of $V \oplus V^{*}$ and compute
$$A_{\ab{~}} = \begin{pmatrix} 0 & \bb 1_{n} \\ \bb 1_{n} & 0 \end{pmatrix} $$
This shows that the canonical pairing $\ab{~}$ is nondegenerate and has signature $(n, n)$.
\marginnote{I could make this into an exercise.}







%%%%%%%%%%%%%%%%%%%%%%%%%%%%%%%%%%%%%%%%%%%%%%%%%%%%%%%%%%%%%%%%%%%%%%%%%%%%%%%%
\section{Symmetries}
As any metric space, we can consider symmetries
$$SO(\vpv) = $$





\lb
The special orthogonal Lie algebra $\so(V \oplus V^{*})$ is identified with
\marginnote{For a metric vector space $(W, g)$ we use here the correspondence between $\so(W, g) ≅ Λ^{2} W$ that identifies
$u ∧ v$ with $u g(v) - v g(u)$ for two elements $u, v ∈ W$.}
$Λ^{2}(\vpv) = Λ^{2}V \oplus V \otimes V^{*} \oplus Λ^{2}V^{*}$.
Elements are possibly of the form
$$β + A + B$$

\lb
We start by investigating $B ∈ Λ^{2}V^{*}$.
Write $B = B_{ij} e^{i} ∧ e^{j}$ in the same basis $e^{1}, \ldots, e^{n}$ as before.
Pick an arbitrary element $X + ξ ∈ \vpv$ and compute
\begin{align*}
    B ∙ (X + ξ) &= B_{ij} e^{j} \ab{e^{i}, X + ξ} - B_{ij}e^{i} \ab{e^{j}, X + ξ} \\
                &= B_{ij} e^{j} e^{i}(X) - B_{ij}e^{i} e^{j} (X) \\
                &= ι_{X} ( B_{ij}e^{i} ∧ e^{j}) \\
                &= ι_{X}B
\end{align*}

\lb
In particular, we notice that $B^{2} = 0$. This lets us compute its exponential
$e^{B} ∈ SO(\vpv)$ as follows
$$ e^{B}(X + ξ) = (1 + B + ½B^{2} + \cdots) (X + ξ) = X + ξ + ι_{X} B $$
This symmetry has the form of a shear transformation and is commonly referred to
as the $B$-field transform.
$$e^{B} = \begin{pmatrix} 1 & 0 \\ B & 1 \end{pmatrix} $$








%%%%%%%%%%%%%%%%%%%%%%%%%%%%%%%%%%%%%%%%%%%%%%%%%%%%%%%%%%%%%%%%%%%%%%%%%%%%%%%%
\section{Isotropic subspaces}




A subspace $U ⊆ V$ in a metric space $(V, g)$ has a complement
$$U┴ = \cb{x ∈ V │ g(x, u) = 0 \tx{ for all } u ∈ U}$$

\lb
A subspace $U ⊆ V$ is \emph{isotropic} if $U ⊆ U ┴$. In other words,
$g \at{U} = 0 $. We call a subspace $U$ \emph{coisotropic} if $U ┴ ⊆ U$.

\marginnote{Exercise: Shat that $U^{*}$ is coisotropic if $U$ is isotropic.}

\lb
For our split signature metric $\ab{~}$
an isotropic subspace $W ∈ \vpv$ can have at most dimension $n$. If it has
dimension $n$, we call it \emph{maximally isotropic} or \emph{Lagrangian}.
\marginnote{This name comes from maximally isotropic subspaces of symplectic
vector spaces $(V, ω)$.}
This is the case if $W$ is both isotropic and coisotropic
\marginnote{Exercise: Shat that $L ⊆ \vpv$ is Lagrangian iff it is isotropic and coisotropic.}




\begin{example}
    $V$ and $V^{*}$ are both maximally isotropic subspaces of $\vpv$.
\end{example}




\begin{example}
    One way to construct a maximally isotropic subspace of $\vpv$ is to start with a
    subspace $U ⊆ V$ and consider
    $$U \oplus Ann(U) ⊆ \vpv$$
    \marginnote{The Annihilator of a subspace is precisely the subspace $Ann(U)$ of $V^{*}$
    of covectors $α$ which restrict to zero on $U$, that is, $α \at{U} = 0$.}
    This subspace is clearly isotropic and of maximal dimension because
    $$ \dim Ann(U) + dim (U) = \dim(V) = n $$
\end{example}




\begin{lemma}
    Every maximal isotropic $L ∈ \vpv$ is of the form $L(E, ε)$
\end{lemma}



\lb
A $B$-field transform $B ∈ Λ^{2}V^{*}$ affects subspaces $L ⊆ \vpv$.





















%%%%%%%%%%%%%%%%%%%%%%%%%%%%%%%%%%%%%%%%%%%%%%%%%%%%%%%%%%%%%%%%%%%%%%%%%%%%%%%%
\section{Spinors}


\lb
Given a metric vector space $(W, g)$,
Spinors arise as a hidden representation of the Lie algebra of special
orthogonal transformations $\so(W, g)$. This does not correspond to a representation
of $\SO(W, g)$ when it is not simply connected.
\marginnote{Reference needed, also suggested reading}
Instead, one constructs the universal cover, called the spin group $Spin(W, g)$ as a certain
set contained in the Clifford algebra $Cl(W, g)$.
\marginnote{More details can be found in wernli2019}

\lb
The important application of spinors in our situation is as a means to encode maximally
isotropic subspaces of $\vpv$.


\begin{definition}
    The \emph{Clifford algebra} of a metric vector space $(W, g)$ is the quotient of the
    tensor algebra of $W$ by the relation $v^{2} = g(v, v) 1$.
\end{definition}
\marginnote{Why is this equivalent to the relation $vw + wv = 2 g(v, w)$? wernli2019}

\lb
General properties of the Clifford algebra.



\begin{definition}
    The Spin group of $(W, g)$ consists of the 
\end{definition}


\lb
The adjoint representation stabilizes $W ん Cl_{W}$.
A vector $v ∈ W$ acts as a reflection across the $g$-orthogonal complement of $v$.

\begin{exercise}
    \begin{enumerate}
        \item Describe the action of an element $uv ∈ Cl_{W}$ on $w ∈ W$.
        \item How is the action of $-uv$ different?
    \end{enumerate}
\end{exercise}

\lb
The conclusion of this exercise should be that the standard representation of
$Spin(W)$ on $W$ is blind to certain subtle differences. We need spinors to detect these
differences.





\lb
To construct spinors for $\V = \vpv$, we can take advantage of the split signature metric.




\lb
Now that we have spinors for $\V$, we can use the following.

\lb
Let $φ ∈ Λ^{●} V^{*}$ be a spinor. Its \emph{null space} $L_{φ}$ is the subspace in $\vpv$
$$L_{φ} = \cb{v ∈ \vpv │ v ∙ φ = 0 }$$
Null spaces of spinors are always isotropic. When $L_{φ}$ is maximally isotropic, we call
the spinor $φ$ \emph{pure}.









\section{Exercises}

\begin{exercise}
    Show that $SO(n)$ is not simply connected for any $n \geq 2$.
    \marginnote{taken from debray2016}
\end{exercise}














%%%%%%%%%%%%%%%%%%%%%%%%%%%%%%%%%%%%%%%%%%%%%%%%%%%%%%%%%%%%%%%%%%%%%%%%%%%%%%%%
%%%%%%%%%%%%%%%%%%%%%%%%%%%%%%%%%%%%%%%%%%%%%%%%%%%%%%%%%%%%%%%%%%%%%%%%%%%%%%%%
%%%%%%%%%%%%%%%%%%%%%%%%%%%%%%%%%%%%%%%%%%%%%%%%%%%%%%%%%%%%%%%%%%%%%%%%%%%%%%%%
\chapter{Exact Courant algebroids}



\lb
The reason we focussed intensively on the linear algebra of $\vpv$ is that it is
precisely the generic fiber of the generalized tangent bundle $TM \oplus T^{*}M$ of
an $n$-manifold $M$. This is the central object of interest in the study of generalized
complex geometry as it is hosting generalized complex structures which we will define in
the next chapter. For now, we will explore the geometric properties of $TM \oplus T^{*}M$
as a Courant algebroid.
We will assume throughout this chapter that $M$ is a smooth manifold.


\section{Lie algebroids}

In fact, we will start with yet another simplification to get acustomed with the world
of *-oids.

\begin{definition}
    A Lie algebroid $(A, [~], \a)$ on $M$ is a vector bundle $A → M$ with a Lie bracket
    $[~]$ on the space of sections $Γ(A)$ and a bundle map $\map{A}[\a]{TM}$ such that
    \begin{enumerate}
        \item conditions
    \end{enumerate}
    \done
\end{definition}


\lb
A Poisson structure $Q$ on $M$ defines a bracket on 1-forms $Ω^{1}_{M}$.
$$[a, b]_{Q} = L_{Q(a)} b - L_{Q(b)} a - d Q(a, b)$$
This is the \emph{Koszul bracket} which is the unique extension of the Poisson bracket
on exact 1-forms that satisfies...


\begin{example}
    \begin{enumerate}
        \item Poisson Lie algebroid
        \item Foliations are Lie algebroids if they are integrable
        \item Atiyah algebroid of a principal bundle
        \item $T_{1,0}X$ of a complex manifold is Lie algebroid on its underlying
            smooth manifold.
    \end{enumerate}
    \done
\end{example}






\section{Courant algebroids}


\lb
Failure to be a Lie algebroid.
Exactness, twists and Severa class.
Symmetries of Courant algebroids.

Examples like

Other higher Courant algebroids like $T \oplus \R \oplus T^{*}$










\chapter{Structures on Courant algebroids}


\lb
Generalized complex structures

\marginnote{hu2006 for Hamiltonian symmetries of a GC structure}
\marginnote{bursztyn2011 for Dirac structures.}
















%%%%%%%%%%%%%%%%%%%%%%%%%%%%%%%%%%%%%%%%%%%%%%%%%%%%%%%%%%%%%%%%%%%%%%%%%%%%%%%%
% The back matter contains appendices, bibliographies, indices, glossaries, etc.
\backmatter
\bibliographystyle{plain}
\bibliography{latex/bib/b}
\printindex
\end{document}
